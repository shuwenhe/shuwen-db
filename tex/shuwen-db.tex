\documentclass[12pt,oneside,a4paper]{ctexbook} %oneside一面,twiside两面
\usepackage[centertags]{amsmath}
\usepackage{amsfonts}
\usepackage{amsthm}
\usepackage{newlfont}
\usepackage{makeidx}
\usepackage{wasysym}
\usepackage{geometry}%页边距和页眉页脚
\usepackage{graphics} %插图
\usepackage{extarrows}
\usepackage{amssymb}%因为所以
\usepackage{titlesec}%TEX标题与正文间距,标题与上下文距离调整
\usepackage{fancyhdr}%页眉
\usepackage{ctex}
\usepackage{listings} % 插入代码

\renewcommand\thesection{\arabic {section}}%去0
\titleformat{\chapter}[display]{\normalfont\huge\bfseries\center}{\chaptertitlename}{1pt}{\Huge}
\titleformat{\section}{\normalfont\Large\bfseries}{\thesection}{1em}{}
\titleformat{\subsection}{\normalfont\large\bfseries}{\thesubsection}{1em}{}
\titleformat{\subsubsection}{\normalfont\normalsize\bfseries}{\thesubsubsection}{1em}{}
\titleformat{\paragraph}[runin]{\normalfont\normalsize\bfseries}{\theparagraph}{1em}{}
\titleformat{\subparagraph}[runin]{\normalfont\normalsize\bfseries}{\thesubparagraph}{1em}{}
\titlespacing*{\chapter} {0pt}{10pt}{10pt}
\titlespacing*{\section} {0pt}{0.5ex plus 1ex minus .2ex}{0.3ex plus .2ex}
\titlespacing*{\subsection} {0pt}{0.25ex plus 1ex minus .1ex}{0.5ex plus .1ex}
\titlespacing*{\subsubsection}{0pt}{3.25ex plus 1ex minus .2ex}{1.5ex plus .2ex}
\titlespacing*{\paragraph} {0pt}{3.25ex plus 1ex minus .2ex}{1em}
\titlespacing*{\subparagraph} {\parindent}{3.25ex plus 1ex minus .2ex}{1em}
\numberwithin{chapter}{part}
\setlength\parskip{\baselineskip}% 增加空行%
\geometry{left=2.0cm,right=2.1cm,top=1.5cm,bottom=2.5cm}% 页边距和页眉页脚
\renewcommand\thesection{\arabic {section}} %目录需要从1-n

\lstset{language=C++}%这条命令可以让LaTeX排版时将C++ 键字突出显示
\lstset{breaklines}%这条命令可以让LaTeX自动将长的代码行换行排版
\lstset{extendedchars=false}%这一条命令可以解决代码跨页时,章节标题,页眉等汉字不显示的问题
\lstset{
    numbers=left,
    numberstyle= \tiny,
    keywordstyle= \color{ blue!70},
    commentstyle= \color{red!50!green!50!blue!50},
    rulesepcolor= \color{ red!20!green!20!blue!20} ,
    escapeinside=``, % 英文分号中可写入中文
    xleftmargin=2em,xrightmargin=2em, aboveskip=1em,
    framexleftmargin=2em,
}

\begin{document}

\pagestyle{fancy}
% \lhead{\includegraphics{shuwenedu.png}}
\chead{diary}
% \rhead{\includegraphics{xstiku.png}}
\lfoot{}
\rfoot{}

日期:\underline{\hbox to 32mm{\renewcommand{\today}{\number\year 年 \number\month 月 \number\day 日}\today}}
\qquad 姓名:\underline{\hbox to 25mm{ShuwenHe}}

\section{20200104}
\begin{enumerate}
\item 
今天妈煮鸡蛋,但是没洗,我很生气,将两个明显带有鸡屎的煮熟的鸡蛋摔个粉碎在地上。
轩轩\&昂昂上来哭闹,不让我,现在我正在北京大学图书馆研究算法时想到此事,感觉确实不应该,
伤害了母亲的心,同时给两个孩子身心健康带来不良影响。

\item 
下午励志是改正发脾气的坏毛病,谁有决绝坏毛病的好的方法吗?请告诉我,做大事的人,一定是遇事
要冷静的人,但是我的那个妈,总是为啥让我会发脾气呢?如何与老妈沟通,但老妈每次还是那样,
想改变一个人的本性是不可能的,除非她本人自己不断学习不断改进,其他没有任何办法。

坚持每天写日记

以后每天到https://www.topcoder.com/做算法题,以后从事算法+数学方向

下午目标:看完go程序设计语言,将书上代码逻辑实现一遍,提交github
时间安排:14:00-18:00,19:00-22:30

每天深入研究一位天才科学家:今日深入研究Gennady Korotkevich,研究Gennady Korotkevich
为何能在算法竞赛领域排名第1名,并记录到people,上传github。

What time to do things, arrange them, and execute them immediately.

Wear watches strictly every day.

Keep a diary in English every day and be strict with yourself.

Fight with the power of a lion, treat others with the heart of a bodhisattva.

Develop github system.

Development of face recognition system

Put English seat first language.
\end{enumerate}

% 
\section{20200428}
\begin{enumerate}
\item 
go后端开发,知道熟练自如开发,再考虑android开发
\end{enumerate}

% 
\section{20200501}
\begin{enumerate}
\item 
集中精力go后端开发
\item 
Brian Kernighan
\item 
AWK
\item 
Chris Lattner Swift 之父 Chris Lattner\\
Chris主要成就:\\
Swift 语言之父,主要作者\\
LLVM 之父,主要作者\\
Clang 主要贡献者\\
我在读博的时候就开始写 LLVM 了。当时 LLVM 是我的博士研究项目,我想把它做成工业界中颠覆性的产品。当时我异想天开,尝试了各种架构设计,想解决以往编译器所有的弊端 -- 结果当然没有如愿。我毕业后,就希望能接着搞 LLVM ,当时只有苹果允许我入职之后继续设计并实现 LLVM 。我想都没想就加入了苹果。\\
直到今天,我还夜以继日、废寝忘食的写代码,我并不是坐那边动动嘴皮子,指挥别人干活的老板。我其实每个周末都在写代码,我很忙的\\
首先,苹果内部所有的项目都不尽相同 -- 工作流程、战略规划、实施细节,到最后发布。Swift 也一样,没有可比性。因为苹果本身就是小组单兵作战模式 -- 每个组负责不同的大方向,组里自己计划和工作,甚至招人都是各自招。\\
并发模型\\

\end{enumerate}

\end{document}